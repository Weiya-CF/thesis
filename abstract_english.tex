\centerline{\LARGE \textbf{Abstract}}

\vspace*{\baselineskip}
Immersive virtual environment can be used to bring both geographically distributed and co-located users to the same virtual place for collaboration. Compared to remote situations, co-located users collaborate in the same virtual world on top of a shared physical workspace. This collocation allows direct user communication and interaction without computer mediation which facilitates collaborative work.

With the development of multi-viewer display and tracking technology, classical projection-based immersive setups (e.g. CAVE) can now support group immersion for co-located users by offering individual stereoscopic views without visual distortion. In this context, the coexistence of information from the virtual and real world, especially when users do not share a common spatial reference frame, provides users with a new kind of perceptual and cognitive experience. We are interested in how users perceive and communicate with each other to achieve a shared context for collaboration, and how we can broaden supported collaborative scenarios with more flexible viewpoint control.

This PhD thesis mainly addresses perceptual and cohabitation issues that we identified in the aim of supporting safe and efficient co-located collaboration in immersive virtual environment. First, we conducted a case study to examine how perceptual conflicts would alter user communication and task performance. Second, we concentrated on the design and evaluation of appropriate navigation paradigms to allow individual virtual navigation while solving cohabitation problems in a shared limited physical workspace. At last, based on the results of previous studies, we designed a generic dynamic model to enable different collaborative modes as well as mode transitions for co-located collaboration in multi-user immersive systems.


\textbf{Keywords}: Virtual Reality, Co-located Collaboration, Navigation, Perception, Cohabitation.

\pagebreak