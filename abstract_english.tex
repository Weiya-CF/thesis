\begin{abstract}[english] %one page
Co-located collaboration in immersive virtual environment.

Virtual reality enables several users located in remote geogra- phical locations to meet themselves in a shared virtual environment to perform a collaborative work such as a simple ob- servation or a co-manipulation of some virtual objects. However, the technical constraints, the use of different material de- vices for each user and the fact of being in a virtual world in- crease the complexity and consequently the misunderstanding between users. This PhD work aims to improve the collabora- tion between users : we propose some new models for desi- gning collaborative virtual environments to deal with the distri- buted architecture, but also to integrate some new metaphors for the collaboration.

To improve the mutual understanding, we have first to ensure that all users have the same state of the virtual environment at the same time. We propose a new way to distribute the data of the virtual environment. This data distribution can be dynami- cally and independently adapted on the network for each virtual object. Thus, a good trade-off between consistency of the virtual environment and responsiveness during interaction can be rea- ched according to the technical constraints, the features of the application, the functions that objects fulfil in the virtual world, and the tasks that users are performing. In order to design a collaborative virtual environment independently from the data distribution and from how the data are displayed to the users (on several material devices), we propose a software architec- ture which clearly separates the data of the virtual environment from the network layer and from the systems for the graphical rendering, the sound rendering, etc. This software architecture is an extension of the PAC model (Presentation, Abstraction, Control) for collaboration applications.

For an efficient collaboration, we have to integrate with adapted metaphors for interaction and collaboration several users using different material devices. Moreover, each user must be able to understand how the others perceive the virtual world and what they can do in this virtual world. We describe a new model, cal- led Immersive Interactive Virtual Cabin, that makes possible to integrate the users in the virtual environment by taking account of their interaction workspaces. This model is based on a hierar- chical structure to organize the different interaction workspaces (visual, sound, haptic, etc.) and on a set of operators to ma- nage and integrate this structure in the virtual environment. In this way, the model can offer new functionalities adapted to the material devices for navigation, interaction and collaboration. It also proposes a structure for representing the interaction capa- bilities of each user in the virtual environment.

This PhD work is a part of the Digiscope project about collaboration for enabling remote experts to examine together some scientific data. In this context, we have run some experimentations to validate the proposed solutions.
\end{abstract}