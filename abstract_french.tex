\centerline{\LARGE \textbf{Résumé}}

\vspace*{\baselineskip}
Les Environnements Virtuels Immersifs (EVI) peuvent être utilisés pour amener des utilisateurs, répartis géographiquement ou co-localisés, à partager un même monde virtuel pour collaborer. Si l’on compare aux situations distantes, les utilisateurs d’une immersion co-localisée collaborent aussi  dans le monde virtuel, mais a contrario, partagent physiquement un même espace de travail. Cette co-localisation facilite le travail collaboratif en permettant des communications directes et des interactions sans médiation informatique entre les utilisateurs.

Avec le développement de l'affichage multi-utilisateur et de la technologie de tracking, les dispositifs immersifs classiques basés sur la rétroprojection (ex. CAVE) peuvent offrir maintenant l'immersion pour plusieurs utilisateurs co-localisés en affichant différentes vues stéréoscopiques sans distorsion visuelle pour chacun d’eux. Dans ce contexte, la coexistence de l'information du monde virtuel et réel, en particulier lorsque les utilisateurs ne partagent pas un référentiel spatial commun, offre aux utilisateurs une nouvelle expérience perceptive et cognitive. Dans cette thèse nous nous sommes intéressés à la façon dont les utilisateurs se perçoivent et communiquent entre eux pour atteindre un contexte commun pour la collaboration, et aux moyens permettant d’élargir des scénarios collaboratifs déjà pris en charge dans ce type de dispositifs, basés sur des techniques de contrôle plus flexible des points de vue des utilisateurs.
 
Cette thèse de doctorat traite donc principalement des problèmes perceptifs et de cohabitation que nous avons identifiés dans l’objectif d’assurer la sécurité et l’efficacité des collaborations co-localisées dans les environnements virtuels immersifs. Tout d'abord, nous avons mené une étude de cas pour examiner comment les conflits perceptifs modifieraient la communication entre les utilisateurs et leur performance. Deuxièmement, nous avons conçu et évalué des paradigmes de navigation appropriés pour permettre la navigation virtuelle individuelle tout en résolvant les problèmes de la cohabitation dans un espace de travail partagé physiquement limité. Enfin, sur la base des résultats de ces travaux, nous avons proposé un modèle dynamique générique qui intègre des contraintes de l'espace de travail physique et aussi ceux du monde virtuel pour gérer la collaboration co-localisée dans les systèmes immersifs multi-utilisateurs.


\textbf{Mots-clef}: Réalité Virtuelle, Collaboration Co-localisée, Navigation, Perception, Cohabitation