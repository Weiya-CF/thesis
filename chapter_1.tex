\chapter{Collaboration in Immersive Virtual Environment}
\label{chapter:context}
\minitoc

\section{Collaboration}
Computers are useful tools for communication and collaboration. With the increasing capacity of network and computing, 

\subsection{CVE}
CVE (Collaborative Virtual Environment) refers to a virtual environment that can be shared by participants connected by a computer (remote or local) network. Users are connected with the virtual world through graphic embodiments called avatars.

\subsection{Remote Collaboration}

\subsection{Co-located Collaboration}


\section{Immersive Virtual Environment - Projection-based Display vs. HMD}

Nowadays, immersive CVEs can mainly be supported by two different display technologies: Projection-based displays, such as workbench, Dome, or CAVE \citep{CruzNeira1993SPV}, and Head-Mounted Displays (HMDs) \citep{Melzer1997HMD}. HMDs can give separate stereoscopic views for each user, and users are visually immersed in the virtual world since the real environment is completely occluded by the screen in front of the eyes. This characteristic has both advantages and inconveniences for co-located collaboration. First of all, users can be co-located either side-by-side or face-to-face with the help of avatars, and they can also talk directly without computer mediation. Additionally, tangible props can be used to improve interaction with virtual content \citep{Salzmann2008TUS}. However, the fact that the perception of physical environment is completely blocked hinders direct communication between users including gestural and postural information. See-through HMDs seem to be a promising option to avoid drawbacks of non-see-through HMD for collaboration between multiple co-located users. For example, in the Studierstube project \citep{Schmalstieg2002Studierstube}, users wear tracked see-through HMDs to perceive both the physical environment and synthetic images at the same time.

Compared with HMDs, projection-based VR displays, especially systems with Immersive Projected Technologies (IPT) like stereoscopic image walls or CAVEs, provide a natural place for teamwork \citep{Johanson2002IWP}, because they are usually shared by a group of people rather than a single user \citep{Benford1996SST}. However, most projection-based displays can only support one fully tracked active user, other co-located users can only share the same point of view with distorted images and participate passively in the collaborative task \citep{Bayon2006Multiple}. This visual distortion impedes users’ collaborative spatial judgments by impacting the perception of depth and direction in the virtual world \citep{Pollock2012Right}.

To avoid visual distortion, different approaches have been explored to provide individual stereoscopic views for multiple co-located users with projection-based display. For example, Simon proposed a multi-viewpoint images technique \citep{Simon2007MVI} which projects different images from multiple viewpoints corresponding to the viewing positions of multiple users, and combines them to a single image on the screen. Users can only get a correct stereoscopic view from predefined position without head tracking, so with static images the interactions that they can have are limited. Other display solutions can also provide individual stereoscopic views like the volumetric display \citep{Grossman2008Volum} and holographic display \citep{Lucente1997Holo}. However, these techniques can only be applied to non-immersive context with limited workspace.


\section{Multi-stereoscopic Projection Display}

In order to have relatively large separate adaptive stereoscopic images for multiple users, time-multiplexing techniques become a working solution. Efforts have been made since 1997, for example, \citet{Agrawala1997TRW} designed a prototype of a responsive workbench to provide two tracked users with independent stereoscopic views without distortion. Later on, \citet{Kunz2002TSC} used a pair of shuttered LCD projectors to generate an active stereo display and then \citet{Frohlich2005MultiViewer} extended this shuttered display to support two to four users with better performance in terms of perceived flicker, brightness of each view and crosstalk. This technique makes it possible for a single screen to support more than one stereoscopic view. Images for different users are separated by active shutter glasses while the separation of the images for the left and right eye is ensured by passive polarized filters. In 2011, \citet{Kulik2011CSS} even developed a projection-based stereoscopic display for six users by using six customized DLP projectors. With each user having an independent stereoscopic view combined with real-time head tracking, more complex collaboration scenarios can be implemented for a group of co-located users.

Besides the development of display technology, various interaction techniques have also been proposed to make co-located collaboration in multi-user projection-based systems easier and more efficient, such as the bent pick ray \citep{Riege2006Bent} and the see-through techniques \citep{Argelaguet2010STT}. However, these interaction techniques focus solely on situations where users are side-by-side. Direct face-to-face collaboration in projection-based display system is supposed to be impossible \citep{Salzmann2009CIC}. Therefore here we tried to study the possibility of having co-located face-to-face collaboration by investigating users' reaction to the perceptual conflicts raised with the dual-presence of the avatar and the real person.

\subsection{Direct Collaboration}

\subsection{Collaboration via Avatars}

\section{Conclusion}

