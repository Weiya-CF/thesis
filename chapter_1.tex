\chapter{Collaboration in Immersive Virtual Environment}
\label{chapter:context}
\minitoc

\section{Introduction}
This chapter presents a general state of the art of different topics related to using immersive virtual environment for collaborative task. Being the joint interest of Virtual Reality (VR) and Computer Supported Collaborative Work (CSCW), research on Shared Virtual Environment (SVE) or Collaborative Virtual Environment becomes more and more popular.

With rapid increasing computation capacity and the development of computer network, multi-agent collaboration via computer-generated virtual space with new types of communication and interaction becomes available.

\section{Virtual Reality}

\subsection{Virtual Environment}
According to \citet{Fox2009Guide}, a virtual environment (VE) is a digital space in which a user's movements are tracked and his or her surroundings rendered, or digitally composed and displayed to the senses, in accordance with those movements.


\subsection{Immersion and Presence}
Immersion describes the extent to which the virtual stimuli takes the place of real world stimuli within the virtuality continuum.
Immersive virtual environment can maintain several sensory modalities such as visual, audio and haptics to give the users the illusion of ``being there" or the feeling of presence.

Studies have shown that immersion, which can invoke the feeling of presence \citep{Slater1994DepthPre}, has not only a pronounced effect on user performance \citep{Dangelo2008Benefits}, but also has an impact on the social relationship between collaborators \citep{Slater2000Small}.

\citet{Slater1994DepthPre} proposed an important distinction between ``presence" and ``Immersion": presence is a subjective phenomenon such as the sensation of being in a virtual environment, while immersion is an objective description of aspects of the system.
Witmer and Singer relate presence in part to the concept of attention: presence may vary across a range of values that depends in part on the allocation of attentional resources.
They think that both involvement and immersion are necessary for experiencing presence.
Another view of presence is based on the ecological theory of perception \citep{Gibson2014Ecological}.
It suggests that the environment offers situated affordances and there is a perception-action coupling: An organism perceives its environment in terms of its affordances, making perception dependent on possible action.




\subsubsection{Device}
\paragraph{Immersive Projection Display}
Projection-based VR displays, especially systems with Immersive Projected Technologies (IPT) like stereoscopic image walls or CAVEs \citep{CruzNeira1993SPV}, 


\paragraph{Head Mounted Display}
Head-Mounted Displays (HMDs) \citep{Melzer1997HMD}, as the name indicates, is a helmet containing two separated displays that are put directly in front of user's eyes allowing visual immersion. 

For example, in the Studierstube project \citep{Schmalstieg2002Stube}, users wear tracked see-through HMDs to perceive both the physical environment and synthetic images at the same time.

\subsubsection{Presence Measurements}
These definitions don't necessarily contradict with each other, but they can have different implications.
In order to refine these definitions, we need better instruments to measure them.
However, the way that presence is measured depends on the theory that is used, so different approaches are used for the measurement of presence.

The most commonly used measures are based on subjective rating through questionnaires.
\citet{Usoh2000Using} have a questionnaire based on their exclusive presence while \citet{Witmer1998MPV} have another one based on attention involvement. \citet{Schuemie2001Pres} make a taxonomy for different measures of presence and a detailed analysis of factors that influence presence in the virtual environment.

All these theories and measurements can help us to study presence for collaborative immersive interactions, so as to answer questions like: how to be present in the immersive remote world? How to manage the awareness of the others (social presence \citep{Mantovani1999Real})? How to manage the affordances of the scene, including human avatars?
The answers to these questions can help us to make a further step towards the realization of a totally immersive CVE which support a higher level of presence.

\subsection{Interaction}
Providing a good interface for interaction is one of the key aspects for the design of new methods and applications in CVEs.
Bowman et al. classify interaction with virtual environments in regard to selection, manipulation, navigation, symbolic input and system control \citep{Bowman2004UIT}.

\subsection{Summary}

\section{Computer Supported Collaborative Work}
CVE (Collaborative Virtual Environment) refers to a virtual environment that can be shared by participants connected by a computer (remote or local) network. Users are connected with the virtual world through graphic embodiments called avatars.

A collaborative virtual environment (CVE) \citep{Benford2001CVE} enables multiple users to interact \citep{Schroeder2006Usability} and to achieve collaborative tasks \citep{Dodds2009Using} within the same virtual environment. Users can be co-located in the same place or connected remotely.

\subsection{Typology}

\subsection{User-related Factors}

\subsubsection{User Representation}
Avatars or virtual humans have long been used as virtual representation of the users, or we can call it user embodiment \citep{Benford1995UEC}. Avatars can provide various important functions in a multi-user virtual environment \citep{Thalmann2001VHR}. For example, a user can get other users' positions, focus of interest and see what they are doing, etc. Users can also get a representation of self in the virtual environment by using self-avatars \citep{Lok2003Effects}, which may be especially useful when working with HMDs. Moreover, according to the task that the users need to achieve in the virtual context, social representation of self may be provided through dedicated virtual clothings and/or virtual tools. In an immersive virtual environment, avatars can play a crucial role for user interaction and communication \citep{Slater1994Body}. Users work together not only in a shared virtual world, but also in a shared social space where social human communication is conveyed by avatars \citep{Roberts2004SSH}.

Several techniques can be used to bring avatars to life. In a desktop configuration, we can use input devices like keyboards to activate predefined animations. While in an immersive virtual environment, real time motion capture systems such as Kinect or optical tracking devices are more suitable to map user's physical motion to an avatar \citep{Mohler2010Effect, Vera2011AugMir, Normand2012FBA}. Due to technology limitations or high cost, humanoid avatars do not commonly support facial expression and other subtle social cues. Another interesting method developed by \citet{Ogi2001SteAva} introduces a fully animated video avatar based on live video capture. \citet{Beck2013IGG} used a similar technique to enable interactions between multiple co-located users and a group of remote users by their avatars.

\subsubsection{User Communication}
Social Human Communication

Spatial Communication (Spatial Reference Frame)

\subsubsection{User Interface}
\paragraph{Co-manipulation}
\paragraph{Co-navigation}

\subsubsection{From Presence to Co-presence}



\subsection{Technical Issues}
Collaborative Virtual Environment

Network Architecture, Scene Management, Data Distribution

\subsection{Summary}


\section{Conclusion}

