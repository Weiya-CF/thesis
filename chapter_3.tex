\chapter{User Cohabitation for Individual Navigation Tasks in Restricted Physical Workspace}
\label{chapter:cohabitation}

\section{Taxonomy}

Navigation is one of the most important and fundamental interaction that we have within the virtual environment.

A lot of navigation techniques of different nature exist to allow users to travel through large virtual space, while physically stay within a limited workspace. These navigation techniques can be classified according to the control law - how a given input from the user can be mapped to a position or velocity change in the virtual world. Typical control laws include position and rate control, and can be implemented by both hardware and software solutions.

The most studied position control is natural walking, which is considered to be the most intuitive way to explore the virtual environment \citep{Ruddle2009BW}. To enable infinite walking in restricted real workspace, one can use both physical locomotion devices like treadmills \citep{Iwata1999Treadmill} and software solutions (e.g. redirection \citep{Peck2008RED}, resetting \citep{Williams2007ELV} and scaling techniques \cite{Interrante2007SLB}). Besides natural walking, walk-in-place \citep{Razzaque2002RWP} and WIM (World-In-Miniature) \citep{Stoakley1995VRW} are also interesting alternatives. Moreover, \citet{Fleury2010Generic} proposed a general model to integrate physical workspace into the virtual world and make the user aware of the physical environment in different ways. \citet{Cirio2012Cube} also summarized several metaphors for safe navigation in a restricted cubic workspace.

Conversely to previous techniques, rate control techniques are based on a virtual vehicle model which enables navigation in large virtual scenes. Users control directly the vehicle's velocity instead of position in the virtual world and can have the sensation of moving (self-motion illusion or vection \citep{Riecke2012Vection}). Actually the virtual vehicle can be controlled by information coming from different sources, for example, various input devices like joystick, haptic arm \citep{Martin2012HDF} or even specific locomotion devices \citep{Marchal2011JOYMAN}. With video cameras or optical tracking systems, users can specify the velocity of the vehicle by motion tracking data of the hand (camera-in-hand \citep{Ware1990EVC}) or head movements \citep{Bourdot2002HCNav}. Gestures \citep{Konrad2003GesturePlay} and postures \citep{Kapri2011Steering} can also be used to move the virtual vehicle. Bowman et al. named this kind of virtual navigation techniques steering metaphors \citep{Bowman2004UIT} which are often relatively easy to implement and can provide efficient and flexible control of virtual navigation.

Some navigation metaphors such as the bubble technique \citep{Dominjon2005Bubble} and the magic barrier tape \citep{Cirio2009MBT} combine both position and rate control in order to enable infinite navigation within restricted real workspace. Position control is used within the physical workspace and then rate control is applied to the virtual vehicle to move further in the virtual world.

\section{6DoF Head Controlled Navigation}

\section{Evaluation}

It's not obvious to compare different navigation techniques without a given virtual context and the associated task. Nevertheless, \citet{Bowman1997TIV} have proposed a list of quality factors which can help us to compare and measure the performance of navigation techniques including general performance indicators such as velocity and accuracy, spatial awareness, ease of learning, ease of use, and level of presence. General performance indicators are relatively easy to measure. We can choose factors that we need depending on the given task and the configuration of the virtual scene.

The level of presence is an important factor for the use of navigation techniques in an immersive virtual environment. \citet{Slater1994DepthPre} consider presence as a subjective phenomenon that can be defined as the sensation of being in a virtual environment, while \citet{Witmer1998MPV} relate presence in part to the concept of attention. These authors believe that both involvement and immersion are necessary for experiencing presence. \citet{Schuemie2001Pres} have proposed a taxonomy of different measures of presence and a detailed analysis of factors that influence presence in the virtual environment.

Another important factor is the user comfort, particularly the lack of symptoms related to cybersickness. \citet{Rich1996AICS} have examined the relationship between the user's level of control of his/her own movements and cybersickness, while \citet{So2001ENS} have carried out experiments to study the influence of the navigation speed (both linear and circular components) on the level of severity of cybersickness. A study by \citet{Stanney2002HPIVE} gives us a better global view that addresses different perceptive factors that are susceptible to cause cybersickness. \citet{LaViola2000DCV} has also discussed about the physiological aspects and theories of cybersickness. To evaluate the level of cybersickness, the Simulator Sickness Questionnaire (SSQ) \citep{Kennedy1993SSQ} remains the reference tool despite the development of various physiological measurements.

\section{Cohabitation}

When it comes to managing multiple users in the same immersive virtual environment for co-located collaboration, most works focused on closely coupled stage situations. In this case, either virtual navigation is disabled (they can just walk inside the tracking area for observation or co-manipulation), or their navigation is controlled by a leader and shared by other users~\citep{Beck2013GGT, Kulik2011CSS}. To enable more complex collaborative scenario including both loosely coupled and closely coupled tasks, it is interesting to allow safe individual navigation for co-located users.