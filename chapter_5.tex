\chapter{A Kinematic Model for Co-located Collaboration}
\label{chapter:kinematic_model}
\minitoc

\section{Introduction}

we designed a kinematic model which integrates real world information into the virtual navigation control in the aim of providing a general framework for co-located collaboration by supporting both consistent and individual modes and also fluent transition between them.


\section{Motivation and Objective}
A generic model: not dependent on system configuration
 
extensible in terms of user number and control method

More flexible bi-directional control of navigation, friction, pseudo-haptic feedback

One user should not be disturbed by other users when in a stable state

Integrating constrains from virtual world into the navigation control.


\section{Basic Components}
\subsection{Kinematic State}
A kinematic state in a 3D space contains three components:
\begin{itemize}
  \item Position and orientation.
  \item Linear and angular velocity.
  \item Linear and angular acceleration.
\end{itemize}

With the Kinematic State (KS) we can store all the information we need to correctly put the vehicle in the virtual world. The KS can be updated either by providing a new acceleration or a new configuration (in form of a 4x4 matrix). For example, we move the vehicle by applying an acceleration which results in a corresponding velocity and a change in configuration. The user also has a KS that is updated by tracking information in a different way than the vehicle.

\subsubsection{Verlet Integration}
The verlet integration is used to update the KS with a new acceleration.

For translation:

\[
\overrightarrow{p_{1}}=\overrightarrow{p_{0}}+\overrightarrow{v_{0}}\Delta t+\frac{1}{2}\overrightarrow{a_{0}}\Delta t^{2}
\]

\[
\overrightarrow{v_{1}}=\overrightarrow{v_{0}}+\frac{\overrightarrow{a_{0}}+\overrightarrow{a_{1}}}{2}\Delta t
\]

And for rotation, we choose to represent the orientation in quaternion for its convenience and simplicity, while angular velocity and acceleration are expressed separately according to euler angles:

\[
\omega_{1}=\omega_{0}+\frac{\dot{\omega_{0}}+\dot{\omega_{1}}}{2}\Delta t
\]

\[
q_{1}=q_{a}q_{v}q_{0}
\]

\[
q_{v}=Q\left(\cos(\frac{\Vert\omega_{0}\Vert\Delta t}{2}),\:\sin(\frac{\Vert\omega_{0}\Vert\Delta t}{2})\frac{\omega_{0}(x)}{\Vert\omega_{0}\Vert\Delta t}),\:\sin(\frac{\Vert\omega_{0}\Vert\Delta t}{2})\frac{\omega_{0}(y)}{\Vert\omega_{0}\Vert\Delta t}),\:\sin(\frac{\Vert\omega_{0}\Vert\Delta t}{2})\frac{\omega_{0}(z)}{\Vert\omega_{0}\Vert\Delta t})\right)
\]

\[
q_{a}=Q\left(\cos(\frac{\Vert\dot{\omega_{0}}\Vert\Delta t}{2}),\:\sin(\frac{\Vert\dot{\omega_{0}}\Vert\Delta t}{2})\frac{\dot{\omega_{0}}(x)}{\Vert\dot{\omega_{0}}\Vert\Delta t}),\:\sin(\frac{\Vert\dot{\omega_{0}}\Vert\Delta t}{2})\frac{\dot{\omega_{0}}(y)}{\Vert\dot{\omega_{0}}\Vert\Delta t}),\:\sin(\frac{\Vert\dot{\omega_{0}}\Vert\Delta t}{2})\frac{\dot{\omega_{0}}(z)}{\Vert\dot{\omega_{0}}\Vert\Delta t})\right)
\]

\subsection{Transfer Function}
t
\subsection{Potential Field}
In the domain of robotics, some algorithms can be used to guide a robot to navigate over a field occupied by some obstacles to get to a goal position, one of them is the potential field function. Generally, 2 kinds of potential fields lead to 2 different behaviors of the robot. An attractive field generates an action vector that points the robot toward the goal, while a repulsive field pushes the robot away from an obstacle.

\subsection{Virtual Constrains}

\section{Structure}

python

class schema

sequential graph

\section{Mode Switching}
state machine

\section{Parameter Setting}


\section{Conclusion}
We need a model of human for reaction of acceleration.