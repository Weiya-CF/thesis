\chapter*{Introduction}
\markboth{\MakeUppercase{Introduction}}{}
%\addcontentsline{toc}{chapter}{Introduction}
%\minitoc
\mtcaddchapter[Introduction]

\section*{Context}

Immersive projection technology, collaborative virtual environment

\section*{Motivation}

\section*{Approach}

\section*{Organization of Thesis}
This thesis contains four chapters and a general introduction and conclusion. The chapters are the following:

Chapter 1 gives a general description of the context and different topics related to using immersive virtual environment for collaborative work.

Chapter 2 first introduces multi-stereoscopic display technology and its application for co-located collaborative work, then describes a user study on the perception of dual-presence during co-located collaboration in a multi-stereoscopic immersive virtual environment (IVE).

Chapter 3 discusses navigation techniques in virtual environments and then proposes a navigation paradigm based on human joystick metaphor for co-located users in a multi-stereoscopic IVE to have individual navigation capability. Several user studies were conducted to evaluate the original human joystick and the altered versions to assess their contributions in term of user cohabitation.

Chapter 4 presents the theoretical concept and implementation of a framework to support collaborative work in multi-stereoscopic IVE. This framework allows both closely-coupled collaboration stage with consistent reference frames between real and virtual worlds, and loosely-coupled collaboration stage with individual navigation control. The automatic or semi-automatic switch is also implemented to allow transitions between these two stages.