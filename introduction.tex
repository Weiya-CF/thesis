\chapter*{Introduction}
\markboth{\MakeUppercase{Introduction}}{}
%\addcontentsline{toc}{chapter}{Introduction}
%\minitoc
\mtcaddchapter[Introduction]

\section*{Context}
Immersive projection technology, collaborative virtual environment

\section*{Issues and Motivation}

\section*{Approach}

\section*{Organization of Thesis}
This thesis contains four chapters and a general introduction and conclusion. The chapters are the following:

Chapter 1 gives a general description of the context and different topics related to using immersive virtual environment for collaborative work.

Chapter 2 introduces different multi-viewer display technology and its application for co-located collaboration, then describes a case study on the perceptual conflicts during co-located collaboration in a multi-stereoscopic immersive virtual environment.

Chapter 3 focuses on searching for solutions to allow independent navigation for co-located users and to manage cohabitation issues in multi-viewer immersive virtual environment. We designed an Altered Human Joystick model and conducted several user studies to evaluate its contributions in term of user cohabitation.

Chapter~\ref{chapter:user_cohab} presents the implementation of Altered Human Joystick along with metrics that we defined to evaluate cohabitation capacity in response to identified cohabitation problems. A series of experiments were carried out to test different alterations to see their influences on user distribution and to find an optimal combination.

Chapter 4 presents the theoretical concept and implementation of a kinematic model to support collaborative work in multi-viewer immersive virtual environment. This model allows both closely-coupled collaboration with spatial collocation between real and virtual worlds, and loosely-coupled collaboration with individual navigation control. The automatic or semi-automatic switch is also implemented to allow transitions between these two stages.

Then the general conclusion summarizes contributions and future work.